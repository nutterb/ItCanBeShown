# The Binomial Theorem

The Binomial Theorem is useful in developing theory around the Binomial and Hypergeometric Distributions.  Two proofs of the Theorem are provided here; one using the traditional approach, and one using a more general approach.  Other useful theorems are provided at the end of this chapter.\\

## Traditional Proof

### Lemma: Pascal's rule

Let $n$ and $x$ be non-negative integers such that $x\leq n$.

Then ${n-1\choose x} + {n-1\choose x-1} = {n\choose x}$.

_Proof:_

$$\begin{align*}
{n-1\choose x} + {n-1\choose x-1}
	&= \frac{(n-1)!}{x!(n-1-x)!} + \frac{(n-1)!}{(x-1)!((n-1)-(x-1))!}\\
  &= \frac{(n-1)!}{x!(n-x-1)!} + \frac{(n-1)!}{(x-1)!(n-1-x+1)!}\\
  &= \frac{(n-1)!}{x!(n-x-1)!} + \frac{(n-1)!}{(x-1)!(n-x)!}\\
  &= \frac{(n-1)!}{x(x-1)!(n-x-1)!} + \frac{(n-1)!}{(x-1)!(n-x)(n-x-1)!}\\
  &= \frac{x(n-1)!}{x(x-1)!(n-x)(n-x-1)!}
	  	+\frac{(n-x)(n-1)!}{x(x-1)!(n-x)(n-x-1)!}\\
  &= \frac{x(n-1)!+(n-x)(n-1)!}{x(x-1)!(n-x)(n-x-1)!} \\
	&= \frac{(x+n-x)(x-1)!}{x(x-1)!(n-x)(n-x-1)!}\\
  &= \frac{n(n-1)!}{x(x-1)!(n-x)(n-x-1)!} \\
	&= \frac{n!}{x!(n-x)!} \\
	&= {n\choose x}
\end{align*}$$

%New Subsection
\subsection{The Binomial Theorem}
\label{BinomialTheor1.2}
Let $a$ and $b$ be constants and let $n$ be any positive integer.  Then\\
$\displaystyle (a+b)^n = \sum\limits_{x=0}^{n} {n\choose x} a^{n-x} b^x$.\\
\\
\it Proof:  \rm \\
\\
This proof is completed by mathematical induction.\\
Base Step: $n=1$\\
$\displaystyle (a+b)^1
	= \sum\limits_{x=0}^{1} {1\choose x} a^{1-x} b^x
	= {1\choose 0} a^{1-0} b^0 + {1\choose 1} a^{1-1} b^1
	= 1\cdot a\cdot 1 + 1\cdot 1\cdot b
	= a+b$\\
\\
Inductive Step: Assume that the Theorem holds for $n$, and show it is true for $n+1$.\\
\\
$\displaystyle (a+b)^{n+1}
	= (a+b)(a+b)^n
	= a(a+b)^n + b(a+b)^n\\\\
\indent = a(a^n + \sum\limits_{x=1}^{n-1}{n\choose x}a^{n-x}b^x + b^n)
		+ b(a^n + \sum\limits_{x=1}^{n-1}{n\choose x}a^{n-x}b^x+b^n)\\\\
\indent = (a^{n+1}+a\sum\limits_{x=1}^{n-1}{n\choose x}a^{n-x}ab^x)
		+ (a^nb+\sum\limits_{x=1}^{n-1}{n\choose x}a^{n-x}b^x+b^{n+1})\\\\
\indent = (a^{n+1}+\sum\limits_{x=1}^{n-1}{n\choose x}a^{n-x+1}ab^x)
		+ (a^nb+\sum\limits_{x=1}^{n-1}{n\choose x}a^{n-x}b^{x+1}+b^{n+1})\\\\
\indent=\footnote[1]{$ab^n={n\choose n}a^{n-n+1}b^n$ 
		which is the term for $x=n$ in the first summation.\\
		\indent\indent $a^nb={n\choose 0}a^{n-0}b^1$
		which is the term for $x=0$ in the second summation.}
	\ \ (a^{n+1}+\sum\limits_{x=1}^{n}a^{n-x+1}b^x)
		+ (\sum\limits_{x=0}^{n-1}{n\choose x}a^{n-x}b^{x+1}+b^{n+1})\\\\
\indent = \footnote[2]{$\sum\limits_{x=0}^{n-1}{n\choose x}a^{n-x}b^{x+1}
	= \sum\limits_{x=1}^{n}{n\choose x-1}a^{n-(x-1)}b^{(x-1)+1}
	= \sum\limits_{x=1}^{n}{n\choose x-1}a^{n-x+1}b^x$}
	\ \ (a^{n+1}+\sum\limits_{x=1}^{n}{n\choose x}a^{n-x+1}b^x)
		+ \sum\limits_{x-1}^{n-1}{n\choose x-1}a^{n-x+1}b^{x+1-1}+b^{n+1})\\
\indent = \footnote[3]{This step is made using Pascal's Rule with $n=n-1$.}
	\ \ a^{n+1} + \sum\limits_{x+1}^{n}{n+1\choose x}a^{n-x+1}b^x + b^{n+1}\\
\indent=a^{n+1}+\sum\limits_{x=1}^{n}{n+1\choose x}a^{(n+1)-x}b^x+b^{n+1}
	=\footnote[4]{$a^{n+1}={n+1\choose 0}a^{(n+1)-0}b^0$
		which is the term for $x=0$ in the summation.\\
		$\indent\indent b^{n+1}={n+1\choose n+1}a^{(n+1)-(n+1)}b^{n+1}$
		which is the term for $x=n+1$ in the summation }
	\ \ \sum\limits_{x=0}^{n+1}{n+1\choose x}a^{(n+1)-x}b^x\\$
This completes both the inductive step and the proof.\rule{.05in}{.05in}\\


%New Section
\pagebreak
\section{General Approach}
\label{BinomialTheor2}
%New Subsection
\subsection{A Binomial Expansion Theorem}
\label{BinomialTheor2.1}
This theorem and its corrolary are provided by Brunette\cite{Bruneta}.\\
\\
For any positive integer $n$, let $B_n = (x_1+y_1) (x_2+y_2) \cdots (x_n+y_n)$.  In the expansion $B_n$, before combining possible like terms, the following are true:
\begin{itemize}
\item[\it i)] There will be $2^n$ terms.
\item[\it ii)] Each of these terms will be a product of $n$ factors.
\item[\it iii)]  In each such product there will be one factor from each binomial (in $B_n$).
\item[\it iv)] Every such product of $n$ factors, one from each binomial, is represented in the expansion.
\end{itemize}
\it Proof: \rm\\
\\
Proof is done by induction.\\
\\
For the case $n=1$, the result is clear.\\
\\
Now assume that the theorem is true for a particulare $n$ and consider $B_{n+1}$.
\[ B_{n+1} = B_n(x_{n+1} + y_{n+1}) = B_nx_{n+1} + B_ny_{n+1} \]
By the inductive assumption, $B_n = T_1 + T_2 + \cdots + T_{2^n}$ where each $T_i$ is a product of $n$ factors, one factor from each binomial.  It follows that every term in the expansion of $B_n+1$ is either of the type $T_ix_{n+1}$ or $T_iy_{n+1}$, for some $1\leq i \leq 2^n$.  But each term of either of the above types is clearly a product of $n+1$ factors with one factor coming from each binomial.  thus, if\it (ii)\rm and \it (iii)\rm are true for $B_n$, then they are true for $B_n+1$.\\
\\
Next, by the inductive assumption, the expansion of $B_n$ is a sum of $2^n+2^n$ terms, i.e., $2^{n+1}$ terms.  This completes the inductive step for \it(i)\rm.\\
\\
Lastly, it remains for us to consider a product of the type $p_1 p_2 \cdots p_n p_{n+1}$ where, for each $1\leq i\leq n+1$, $p_i = x_i$ or $p_i = y_i$.  by the inductive hypothesis, $p_1 p_2 \cdots p_n$ is a terms in the expansion of $B_n$.  If $p_{n+1} = x_{n+1}$, then $p_1 p_2 \cdots p_n p_{n+1}$ is a term in the expansion of $B_nx_{n+1}$, and so of $B_{n+1}$.  Likewise, if $p_{n+1}=y_{n+1}$, then $p_1 p_2 \cdots p_n p_{n+1}$ is a term in the expansion of $B_n y_{n+1}$, and so of $B_{n+1}$.  This completes the inductive step and the proof.\rule{.05in}{.05in}\\


%New Subsection
\newpage
\subsection{Corollary: Binomial Theorem}
\label{BinomialTheor2.2}
Let $x$ and $y$ be constants and let $n$ be any positive integer.\\
Then $\displaystyle (x+y)^n = \sum\limits_{i=0}^{n} {n\choose i} x^{n-i} y^i\\$
\\
\it Proof: \rm \\
\\
Since each term in the expansion will have $n$ terms, each term must follow the form $x^{n-i} y^i$ for $0 \leq i \leq n$, and in all, there are $2^n$ such terms.  For any given value of $i$, the number of terms of the form $x^{n-i}y^i$ is clearly the number of ways one can choose the $i$ factors of $y$ from the $n$ available binomials, i.e., ${n\choose i}$, which gives\\
\\
$\displaystyle (x+y)^n = \sum\limits_{i=0}^{n}{n\choose i} x^{n-i} y^i$ \rule{.05in}{.05in}

%New Section
\section{Other Theorems}
\label{BinomialTheor3}
%New Subsection
\subsection{Theorem}
\label{BinomialTheor3.1}
$\displaystyle {N_1\choose 0}{N_2\choose n} + {N_1\choose 2}{N_2\choose n-1} + \cdots
		+ {N_1\choose n-1}{N_2\choose 1} + {N_1\choose n}{N_2\choose 0}
	= {N_1+N_2\choose n}$\\
\\
where $0 \leq n \leq N_1 + N_2$.\\
\\
\it Proof:\rm\\
Using the Binomial Theorem we establish\\
$\displaystyle (1+a)^{N-1} (1+a)^{N_2} = (1+a)^{N_1+N_2}\\\\
\indent\Rightarrow [{N_1\choose 0}a^0+\cdots+{N_1\choose N_1}a^{N_1}]\cdot
		[{N_2\choose 0}a^0+\cdots+{N_2\choose N_2}a^{N_2}]\\\\
\indent\indent={N_1+N_2\choose 0}+{N_1+N_2\choose 1}a+\cdots
		+{N_1+N_2\choose N_1+N_2}a^{N_1+N_2}\\\\$
Expanding the left side of the equation gives\\
\\
$\displaystyle {N_1\choose 0}{N_2\choose 0} + {N_1\choose 0}{N_2\choose 1}a + \cdots
		+ {N_1\choose 0}{N_2\choose N_2}a^{N_2} + {N_1\choose 1}{N_2\choose 0}a\\\\
\indent\indent + \cdots + {N_1\choose 1}{N_2\choose N_2}a^{N_2+1}
		+ \cdots + {N_1\choose N_1}{N_2\choose 0}a^{N_1}
		+ {N_1\choose N_1}{N_2\choose 1}a^{N_1+1}\\\\
\indent\indent + \cdots + {N_1\choose N_1}{N_2\choose N_2}a^{N_1+N_2}\\\\
$\newpage\noindent$
\displaystyle\indent = {N_1\choose 0}{N_2\choose 0}+{N_1\choose 0}{N_2\choose 1}a
		+ {N_1\choose 1}{N_2\choose 0}a\\\\
\indent		+ {N_1\choose 0}{N_2\choose 2}a^2+{N_1\choose 1}{N_2\choose 1}a^2
		+ {N_1\choose 2}{N_2\choose 0}a^2\\\\
\indent\indent	+ \cdots + {N_1\choose N_1}{N_2\choose N_2}a^{N_1+N_2}\\$
\\
Notice that for any $n$ where $0 \leq n \leq N_1 + N_2$, the coefficient for $a^n$, found by combining like terms, is ${N_1\choose 0}{N_2\choose n} + {N_1\choose 1}{N_2\choose n-1} + \cdots+{N_1\choose n-1}{N_2\choose 1} + {N_1\choose 0}{N_2\choose n}$ and, by the equivalence of the first equation in the proof, is equal to the coefficient ${N_1 + N_2\choose n}$. \rule{.05in}{.05in} \\

%New Subsection
\subsection{Theorem}
\label{BinomialTheor3.2}
$\displaystyle \frac{\sum\limits_{i=1}^{n}{N_1\choose i}{N_2\choose n-i}}{{N_1+N_2\choose n}} = 1$ for $0 \leq n \leq N_1 + N_2$.\\
\\
\it Proof: \rm \\
\\
Theorem \ref{BinomialTheor3.1} establishes the equality \\
\\
$\displaystyle {N_1\choose 0}{N_2\choose n}+{N_1\choose 2}{N_2\choose n-1} + \cdots
		+ {N_1\choose n-1}{N_2\choose 1}+{N_1\choose n}{N_2\choose 0}
	= {N_1+N_2\choose n}\\\\
\indent\Rightarrow\sum\limits_{i=1}^{n}{N_1\choose i}{N_2\choose n-i}
	= {N_1+N_2\choose n}\\\\
\indent\Rightarrow\frac{\sum\limits_{i=1}^{n} {N_1\choose i}{N_2\choose n-i}} {{N_1+N_2\choose n}}
	= 1$ \rule{.05in}{.05in}\\
